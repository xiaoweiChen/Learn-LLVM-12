实际的编译器必须能够处理许多文件。通常,开发人员用主编译单元的名称来调用编译器。这个编译单元可以引用其他文件,例如:通过C语言中的\#include指令,或Python或Modula-2中的import语句。导入的模块可以导入其他模块等。所有这些文件都必须加载到内存中,并通过编译器的分析阶段运行。在开发过程中,开发人员可能会犯语法或语义错误。当检测到错误信息时,应该打印出错误信息,包括源行和一个标记,很明显,这个基本组件很重要。\par

幸运的是,LLVM附带了一个解决方案:LLVM::SourceMgr类。通过调用AddNewSourceBuffer()方法,将新源文件添加到SourceMgr。或者,通过调用AddIncludeFile()来加载文件。这两个方法都返回一个ID来标识缓冲区,可以使用这个ID来检索指向关联文件的内存缓冲区的指针。要在文件中定义位置,必须使用llvm::SMLoc类,这个类将指针封装到缓冲区中。各种PrintMessage()会向用户发出错误和其他信息消息。\par

只缺少一种集中定义消息的方法。大型软件(如编译器)中,您不希望到处都是消息字符串。如果有人要求更改消息或将其翻译成另一种语言,那么最好把它们放在中心位置!\par

一种简单的方法是,每个消息都有一个ID(一个enum成员)、一个严重性级别和一个包含消息的字符串。代码中,只引用消息ID。严重性级别和消息字符串仅在打印消息时使用。必须一致地管理这三个项(ID、安全级别和消息)。LLVM库使用一个预处理器来解决这个问题。数据存储在一个后缀为a.def的文件中,并封装在一个宏名称中。该文件通常包含多次,宏的定义不同。它的定义在include/tinylang/Basic/Diagnostic.def文件路径中,如下所示:\par

\begin{lstlisting}[caption={}]
#ifndef DIAG
#define DIAG(ID, Level, Msg)
#endif

DIAG(err_sym_declared, Error, "symbol {0} already declared")
#undef DIAG
\end{lstlisting}

第一个宏参数ID是枚举标签,第二个参数Level是严重性,第三个参数Msg是消息文本。这样,就可以定义一个DiagnosticsEngine类发出的错误消息。该接口位于include/tinylang/Basic/Diagnostic.h文件中:\par

\begin{lstlisting}[caption={}]
#ifndef TINYLANG_BASIC_DIAGNOSTIC_H
#define TINYLANG_BASIC_DIAGNOSTIC_H

#include "tinylang/Basic/LLVM.h"
#include "llvm/ADT/StringRef.h"
#include "llvm/Support/FormatVariadic.h"

#include "llvm/Support/SMLoc.h"
#include "llvm/Support/SourceMgr.h"
#include "llvm/Support/raw_ostream.h"
#include <utility>

namespace tinylang {
\end{lstlisting}

包含必要的头文件之后,现在使用Diagnostic.def来定义枚举。为了不污染全局命名空间,需要使用命名空间diag:\par

\begin{lstlisting}[caption={}]
namespace diag {
	enum {
		#define DIAG(ID, Level, Msg) ID,
		#include "tinylang/Basic/Diagnostic.def"
	};
} // namespace diag
\end{lstlisting}

DiagnosticsEngine类使用SourceMgr实例通过report()发出消息,消息可以有参数。要实现这个功能,必须使用LLVM的可变格式支持。在静态方法的帮助下检索消息文本和严重性级别。此外,还会计算发出的错误消息的数量:\par

\begin{lstlisting}[caption={}]
class DiagnosticsEngine {
	static const char *getDiagnosticText(unsigned DiagID);
	static SourceMgr::DiagKind
	getDiagnosticKind(unsigned DiagID);
\end{lstlisting}

消息字符串由getDiagnosticText()返回,级别由getDiagnosticKind()返回。这两种方法都将在以后的.cpp文件中实现:\par

\begin{lstlisting}[caption={}]
	SourceMgr &SrcMgr;
	unsigned NumErrors;
	
public:
	DiagnosticsEngine(SourceMgr &SrcMgr)
		: SrcMgr(SrcMgr), NumErrors(0) {}
		
	unsigned nunErrors() { return NumErrors; }
\end{lstlisting}

因为消息可以有可变数量的参数,所以C++中的解决方案是使用可变参数模板。当然,LLVM提供的formatv()函数也使用这种方法。要获得格式化的消息,我们只需要转发模板参数:\par

\begin{lstlisting}[caption={}]
	template <typename... Args>
	void report(SMLoc Loc, unsigned DiagID,
				Args &&... Arguments) {
		std::string Msg =
			llvm::formatv(getDiagnosticText(DiagID),
						  std::forward<Args>(Arguments)...)
				.str();
		SourceMgr::DiagKind Kind = getDiagnosticKind(DiagID);
		SrcMgr.PrintMessage(Loc, Kind, Msg);
		NumErrors += (Kind == SourceMgr::DK_Error);
	}
};

} // namespace tinylang

#endif
\end{lstlisting}

这样,就实现了类的大部分内容。只缺少getDiagnosticText()和getDiagnosticKind()。它们在lib/Basic/Diagnostic.cpp文件中定义,并使用Diagnostic.def文件:\par

\begin{lstlisting}[caption={}]
#include "tinylang/Basic/Diagnostic.h"
using namespace tinylang;
namespace {
	const char *DiagnosticText[] = {
		#define DIAG(ID, Level, Msg) Msg,
		#include "tinylang/Basic/Diagnostic.def"
	};
\end{lstlisting}

与头文件中一样,DIAG宏定义为检索所需的部分。这里,将定义一个数组来保存文本消息。因此,DIAG宏只返回Msg部分。我们将使用相同的方法:\par

\begin{lstlisting}[caption={}]
SourceMgr::DiagKind DiagnosticKind[] = {
#define DIAG(ID, Level, Msg) SourceMgr::DK_##Level,
#include "tinylang/Basic/Diagnostic.def"
};
} // namespace
\end{lstlisting}

不出所料,这两个函数都只是简单地对数组进行索引,并返回所需的数据:\par

\begin{lstlisting}[caption={}]
const char *
DiagnosticsEngine::getDiagnosticText(unsigned DiagID) {
	return DiagnosticText[DiagID];
}

SourceMgr::DiagKind
DiagnosticsEngine::getDiagnosticKind(unsigned DiagID) {
	return DiagnosticKind[DiagID];
}
\end{lstlisting}

SourceMgr和DiagnosticsEngine类的这种组合为其他组件提供了良好的基础。让我们先在词法分析器中使用一下!\par




















































