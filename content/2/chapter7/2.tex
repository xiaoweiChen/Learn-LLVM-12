LLVM IR中的异常处理与平台的支持密切相关。这里,我们将了解libunwind中最常见的异常处理方式。它能在C++中使用了,所以我们首先看一个C++中的例子,其中bar()函数抛出一个int或double值,如下所示:\par

\begin{lstlisting}[caption={}]
int bar(int x) {
	if (x == 1) throw 1;
	if (x == 2) throw 42.0;
	return x;
}
\end{lstlisting}

foo()函数调用bar(),但只处理抛出的int值。foo()函数声明它只抛出int值,如下所示:\par

\begin{lstlisting}[caption={}]
int foo(int x) throw(int) {
	int y = 0;
	try {
		y = bar(x);
	}
	catch (int e) {
		y = e;
	}
	return y;
}
\end{lstlisting}

抛出异常需要对运行时库进行两次调用。首先,调用\underline{~~}cxa\underline{~}allocate\underline{~}exception()为异常分配内存,这个函数以要分配的字节数作为参数,异常(示例中的int或double值)会复制到分配的内存中,然后调用\underline{~~}cxa\underline{~}throw()引发异常。这个函数有三个参数:指向已分配异常的指针;关于异常类型信息;以及指向析构函数的指针(如果异常有效负载有的话)。函数的作用是:\underline{~~}cxa\underline{~}throw()函数启动堆栈展开过程,并且不返回。在LLVM IR中,对int值做如下处理:\par

\begin{tcolorbox}[colback=white,colframe=black]
\%eh = tail call i8* @\underline{~~}cxa\underline{~}allocate\underline{~}exception(i64 4) \\
\%payload = bitcast i8* \%eh to i32* \\
store i32 1, i32* \%payload \\
tail call void @\underline{~~}cxa\underline{~}throw(i8* \%eh, \\
\hspace*{3cm}i8* bitcast (i8** @\underline{~}ZTIi to i8*), i8* \\
\hspace*{3cm}null) \\
unreachable
\end{tcolorbox}

\underline{~}ZTIi是描述int类型的信息。对于双精度类型,使用的是\underline{~}ZTId。对\underline{~~}cxa\underline{~}throw()的调用标记在尾部,因为它是这个函数中的最后一个调用,或许可以重用当前堆栈帧。\par

到目前为止,还没有针对llvm进行任何操作。变化会发生在foo()函数中,因为对bar()的调用可能会引发异常。如果是int类型异常,则必须将控制流转移到catch子句的IR代码上。为了实现这一点,必须使用invoke 调用指令而不是call调用指令,具体的方式如下面的代码所示:\par

\begin{tcolorbox}[colback=white,colframe=black]
\%y = invoke i32 @\underline{~}Z3bari(i32 \%x) to label \%next \\
\hspace*{6cm}unwind label \%lpad
\end{tcolorbox}

这两个指令之间的区别在于invoke调用有两个关联的标签。第一个标签是调用函数正常结束后继续执行的地方,通常是ret指令。在之前的示例中,这个标签称为\%next。如果产生异常,则在带有\%lpad标签,所谓的“着陆垫”上继续执行。\par

“着陆垫”是一个基本模块,必须以landingpad指令开始,着陆指令向LLVM提供关于已处理异常类型的信息。对于foo()函数,给出了以下信息:\par

\begin{tcolorbox}[colback=white,colframe=black]
lpad: \\
\%exc = landingpad { i8*, i32 } \\
\hspace*{3cm}cleanup \\
\hspace*{3cm}catch i8* bitcast (i8** @\underline{~}ZTIi to i8*) \\
\hspace*{3cm}filter [1 x i8*] [i8* bitcast (i8** @\underline{~}ZTIi to  \\
\hspace*{3.5cm}i8*)]
\end{tcolorbox}

这里有三种可能的行动类型,概述如下:\par

\begin{enumerate}
\item cleanup: 这表示清除当前状态的代码。通常,这用于调用局部对象的析构函数。如果有此标记,则在堆栈展开期间始终调用着陆垫。

\item catch: 这是一个类型-值对列表,表示可以处理的异常类型。如果在此列表中发现抛出的异常类型,则调用着陆垫。在foo()函数中,值是指向int类型的C++运行时的指针,类似于\underline{~~}cxa\underline{~}throw()函数的形参。
	
\item filter: 指定异常类型数组。如果在数组中找不到当前异常的异常类型,则调用着陆垫。这用于实现throw()的规范,对于foo()函数,数组只有一个成员——int类型的类型信息。
\end{enumerate}

着陆指令的结果类型是一个\{i8*, i32 \}结构。第一个元素是指向抛出异常的指针,而第二个元素是类型选择器。让我们从结构中提取这两个元素:\par

\begin{tcolorbox}[colback=white,colframe=black]
\%exc.ptr = extractvalue \{ i8*, i32 \} \%exc, 0 \\
\%exc.sel = extractvalue \{ i8*, i32 \} \%exc, 1
\end{tcolorbox}

使用数字对类型进行选择,它可以帮助我们确定为什么要使用着陆垫。如果当前异常类型与着陆指令catch部分中给出的异常类型匹配,则它具有正值。如果当前异常类型不匹配过滤器中的任何值,则该值为负值,如果应该调用清理代码,则该值为0。\par

基本上,类型选择器是类型信息表中偏移信息,该表由landingpad指令的catch和filter中给定的值构造。在优化过程中,多个着陆垫可以组合成一个,这意味着该表格的结构在IR层面上是未知的。要检索给定类型的类型选择器,需要调用@llvm.eh.typeid.for指令函数。我们需要检查类型选择器的值是否对应int的类型信息,以便执行catch (int e) \{\}块中的代码:\par

\begin{tcolorbox}[colback=white,colframe=black]
\%tid.int = tail call i32 @llvm.eh.typeid.for( \\
\hspace*{6cm}i8* bitcast (i8** @\underline{~}ZTIi to  \\
\hspace*{6cm}i8*)) \\
\%tst.int = icmp eq i32 \%exc.sel, \%tid.int \\
br i1 \% tst.int, label \%catchint, label \%filterorcleanup
\end{tcolorbox}

异常的处理是通过调用\underline{~~}cxa\underline{~}begin\underline{~}catch()和\underline{~}cxa\underline{~}end\underline{~}catch()来实现的。\underline{~~}cxa\underline{~}begin\underline{~}catch()函数需要一个参数:当前异常。这是着陆指令返回的值之一。它返回一个指向异常的指针——我们的例子中是一个int值。函数的作用是:\underline{~}cxa\underline{~}end\underline{~}catch()标志着异常处理的结束,并释放由\underline{~}cxa\underline{~}allocate\underline{~}ex\allowbreak ception()分配的内存。请注意,如果在catch块中抛出另一个异常,则运行时行为会复杂很多。异常处理如下:\par

\begin{tcolorbox}[colback=white,colframe=black]
catchint: \\
\%payload = tail call i8* @\underline{~~}cxa\underline{~}begin\underline{~}catch(i8* \%exc.ptr) \\
\%payload.int = bitcast i8* \%payload to i32* \\
\%retval = load i32, i32* \%payload.int \\
tail call void @\underline{~~}cxa\underline{~}end\underline{~}catch() \\
br label \%return
\end{tcolorbox}

如果当前异常类型与throws()声明中的列表不匹配,则调用“意外”的异常处理程序。首先,我们需要再次检查类型选择器,如下所示:\par

\begin{tcolorbox}[colback=white,colframe=black]
filterorcleanup: \\
\%tst.blzero = icmp slt i32 \%exc.sel, 0 \\
br i1 \%tst.blzero, label \%filter, label \%cleanup
\end{tcolorbox}

如果类型选择器的值小于0,则调用处理程序:

\begin{tcolorbox}[colback=white,colframe=black]
filter: \\
tail call void @\underline{~~}cxa\underline{~}call\underline{~}unexpected(i8* \%exc.ptr) \#4 \\
unreachable
\end{tcolorbox}

同样,处理程序不会进行返回。\par

在这种情况下不需要清理工作,所以清理代码所做的全部工作就是继续执行堆栈展开器(stack unwinder):\par

\begin{tcolorbox}[colback=white,colframe=black]
cleanup: \\
resume { i8*, i32 } \%exc
\end{tcolorbox}

还有一点还不清楚:libunwind驱动堆栈展开,但它并没有绑定到语言,语言依赖处理是在函数中完成的。对于Linux上的C++,相应函数称为\underline{~~}gxx\underline{~}personality\underline{~}v0()。根据平台或编译器的不同,这个名称可能有所不同。每一个需要参与堆栈展开的功能都附加了一个功能,相应函数分析函数是否捕获异常、是否有不匹配的筛选列表或是否需要清理调用。它将这些信息返回给展开器,展开器相应地进行操作。在LLVM IR中,函数的指针作为函数定义的一部分给出,如下代码片段所示:\par

\begin{tcolorbox}[colback=white,colframe=black]
define i32 @\underline{~}Z3fooi(i32) personality i8* bitcast \\
\hspace*{3cm}(i32 (...)* @\underline{~~}gxx\underline{~}personality\underline{~}v0 to \\
\hspace*{3cm}i8*)
\end{tcolorbox}

这样,异常处理功能就完成了。\par

要在编程语言的编译器中使用异常处理,最简单的策略是利用现有的C++运行时函数。这样做还有一个好处,即可以与C++互操作。缺点是将一些C++运行时绑定到您的语言的运行时中——最明显的是内存管理。如果想避免这种情况,需要创建自己的\underline{~}cxa\underline{~}函数,或等效函数。不过,还是可以继续使用libunwind,因为提供了堆栈展开的机制。\par

\begin{enumerate}
\item 了解一下如何创建IR。我们在第3章中创建了calc表达式编译器。现在,我们将扩展表达式编译器的代码生成器,以便在执行除0时处理异常。生成的IR将检查一个除数的除数是否为0,如果为true,则会引发异常。我们还将为该函数添加一个着陆垫,可以捕获异常,在控制台打印“Divide by zero!”,最后结束计算。这种简单的情况下,使用异常处理并不是那么必要,但它可以让我们集中精力于代码生成。我们将所有代码添加到CodeGenerator.cpp文件中。首先,添加所需的新字段和一些帮助器方法,需要存储\underline{~}cxa\underline{~}allocate\underline{~}exception()和\underline{~}cxa\underline{~}throw()函数的LLVM声明,包括函数类型和函数本身,需要GlobalVariable实例来保存类型信息。我们还需要引用包含着陆台的基本块和只包含不可达指令的基本块:
\begin{lstlisting}[caption={}]
GlobalVariable *TypeInfo = nullptr;
FunctionType *AllocEHFty = nullptr;
Function *AllocEHFn = nullptr;
FunctionType *ThrowEHFty = nullptr;
Function *ThrowEHFn = nullptr;
BasicBlock *LPadBB = nullptr;
BasicBlock *UnreachableBB = nullptr;
\end{lstlisting}

\item 我们还添加了一个新的辅助函数来创建用于比较两个值的IR。createICmpEq()函数接受左值和右值作为参数进行比较,它创建一个比较指令,用于测试值的相等性,并创建一个分支指令,用于相等和不相等情况下的两个基本块。这两个基本块通过TrueDest和FalseDest参数中的引用返回。新基本块的标签可以在TrueLabel和FalseLabel参数中给出:
\begin{lstlisting}[caption={}]
void createICmpEq(Value *Left, Value *Right,
		BasicBlock *&TrueDest,
		BasicBlock *&FalseDest,
		const Twine &TrueLabel = "",
		const Twine &FalseLabel = "") {
	Function *Fn =
		Builder.GetInsertBlock()->getParent();
	TrueDest = BasicBlock::Create(M->getContext(),
								  TrueLabel, Fn);
	FalseDest = BasicBlock::Create(M->getContext(),
								  FalseLabel, Fn);
	Value *Cmp = Builder.CreateCmp(CmpInst::ICMP_EQ,
								  Left, Right);
	Builder.CreateCondBr(Cmp, TrueDest, FalseDest);
}
\end{lstlisting}

\item 要使用运行时中的函数,我们需要创建几个函数声明。在LLVM中,必须构造给出签名的函数类型(以及函数本身)。我们使用createFunc()方法来创建这两个对象。函数需要引用FunctionType和Function指针、新声明函数的名称和结果类型。参数类型列表是可选的,表示变量参数列表的标志设置为false,表示参数列表中没有变量部分:
\begin{lstlisting}[caption={}]
void createFunc(FunctionType *&Fty, Function *&Fn,
					const Twine &N, Type *Result,
					ArrayRef<Type *> Params = None,
					bool IsVarArgs = false) {
	Fty = FunctionType::get(Result, Params, IsVarArgs);
	Fn = Function::Create(
		Fty, GlobalValue::ExternalLinkage, N, M);
}
\end{lstlisting}
\end{enumerate}

这些准备工作完成后,我们继续生成IR来引发异常。\par

\hspace*{\fill} \par %插入空行
\textbf{引发异常}

为了生成引发异常的IR代码,添加了一个addThrow()方法。这个新方法需要初始化新字段,然后生成IR,通过\underline{~~}cxa\underline{~}函数引发异常。所引发异常的有效负载是int类型的,可以设置为任意值。下面是我们需要编码的内容:\par

\begin{enumerate}
\item 新的addThrow()方法首先检查TypeInfo字段是否已初始化。如果不是,则创建一个i8*类型的全局外部常量和一个\underline{~}ZTIi名称。它表示描述C++ int类型的C++元数据:
\begin{lstlisting}[caption={}]
void addThrow(int PayloadVal) {
	if (!TypeInfo) {
		TypeInfo = new GlobalVariable(
			*M, Int8PtrTy,
			/*isConstant=*/true,
			GlobalValue::ExternalLinkage,
			/*Initializer=*/nullptr, "_ZTIi");
\end{lstlisting}

\item 初始化继续使用createFunc()辅助方法为\underline{~}cxa\underline{~}allocate\underline{~}exception()和\underline{~}cxa\underline{~}throw functions()创建IR声明:
\begin{lstlisting}[caption={}]
	createFunc(AllocEHFty, AllocEHFn,
				"__cxa_allocate_exception", 
				Int8PtrTy,
				{Int64Ty});
	createFunc(ThrowEHFty, ThrowEHFn, "__cxa_throw",
				VoidTy,
				{Int8PtrTy, Int8PtrTy, Int8PtrTy});
\end{lstlisting}

\item 使用异常处理的函数需要一个特殊函数,它有助于堆栈展开。我们添加IR代码来声明C++库中的\underline{~}gxx\underline{~}personality\underline{~}v0()函数,并将其设置为当前特殊函数。当前函数不存储为字段,但可以使用Builder实例来查询当前的基本块,它将函数存储为父字段:
\begin{lstlisting}[caption={}]
	FunctionType *PersFty;
	Function *PersFn;
	createFunc(PersFty, PersFn,
				"__gxx_personality_v0", Int32Ty, None,
				true);
	Function *Fn =
		Builder.GetInsertBlock()->getParent();
	Fn->setPersonalityFn(PersFn);
\end{lstlisting}

\item 接下来,创建并填充着陆垫的基本块。首先,我们需要保存指向当前基本块的指针。然后,创建一个新的基本块,在构建器中设置它作为插入指令的基本块,并调用addLandingPad()方法。此方法生成用于处理异常的IR代码,将在下一节捕获异常中进行描述。下面的代码填充了着陆垫的基本块:
\begin{lstlisting}[caption={}]
	BasicBlock *SaveBB = Builder.GetInsertBlock();
	LPadBB = BasicBlock::Create(M->getContext(),
								"lpad", Fn);
	Builder.SetInsertPoint(LPadBB);
	addLandingPad();
\end{lstlisting}

\item 初始化部分已经完成创建保存不可达指令的基本块。同样,我们创建了一个基本块,并将其设置为构建器上的插入点。然后,向它添加一个不可访问的指令。最后,将构建器的插入点设置回已保存的SaveBB实例,以便将以下IR添加到正确的基本块中:
\begin{lstlisting}[caption={}]
	UnreachableBB = BasicBlock::Create(
		M->getContext(), "unreachable", Fn);
	Builder.SetInsertPoint(UnreachableBB);
	Builder.CreateUnreachable();
	Builder.SetInsertPoint(SaveBB);
}
\end{lstlisting}

\item 要引发异常,需要通过调用\underline{~~}cxa\underline{~}allocate\underline{~}exception()函数来为异常和负载分配内存。我们的异常是C++的int类型,其大小通常为4字节。我们为其创建一个常量unsigned值,并将其作为参数调用函数。函数类型和函数声明已经初始化,所以只需要创建一个调用指令:
\begin{lstlisting}[caption={}]
	Constant *PayloadSz =
		ConstantInt::get(Int64Ty, 4, false);
	CallInst *EH = Builder.CreateCall(
		AllocEHFty, AllocEHFn, {PayloadSz});
\end{lstlisting}

\item 接下来,我们将PayloadVal值存储到分配的内存中。为此,需要通过调用ConstantInt::get()函数来创建一个LLVM IR常量。指向分配内存的指针是i8*类型的,但是要存储i32类型的值,需要创建一个bitcast指令来强制转换该类型:
\begin{lstlisting}[caption={}]
	Value *PayloadPtr =
		Builder.CreateBitCast(EH, Int32PtrTy);
	Builder.CreateStore(
		ConstantInt::get(Int32Ty, PayloadVal, true),
		PayloadPtr);
\end{lstlisting}

\item 最后,通过调用\underline{~~}cxa\underline{~}抛出函数引发异常。因为这个函数实际上会在处理函数中引发异常,所以我们需要使用invoke指令而不是call指令。与call指令不同,invoke指令结束一个基本块,因为它有两个后续基本块。这里,是UnreachableBB和LPadBB基本块。如果函数没有引发异常,控制流则转移到UnreachableBB基本块。由于\underline{~~}cxa\underline{~}throw()函数的设计,这种情况永远不会发生。控制流会传输到LPadBB基本块来处理异常。这样就完成了addThrow()方法的实现:
\begin{lstlisting}[caption={}]
	Builder.CreateInvoke(
		ThrowEHFty, ThrowEHFn, UnreachableBB, LPadBB,
		{EH, ConstantExpr::getBitCast(TypeInfo, 
		Int8PtrTy),
		ConstantPointerNull::get(Int8PtrTy)});
}
\end{lstlisting}
\end{enumerate}

接下来,我们添加代码来生成处理异常的IR。\par

\hspace*{\fill} \par %插入空行
\textbf{捕获异常}

为了生成IR代码来捕获异常,添加了一个addLandingPad()方法。生成的IR从异常中提取类型信息。如果匹配C++的int类型,则通过输出“Divide by zero!”,并从函数返回。如果类型不匹配,则只需执行一个resume指令,它将控制权转移回运行时。因为在调用层次结构中没有其他函数来处理此异常,所以运行时将终止应用程序。以下是生成IR来捕获异常需要的步骤:\par

\begin{enumerate}
\item 生成的IR中,需要调用C++运行库中的\underline{~}cxa\underline{~}begin\underline{~}catch()和\underline{~}cxa\underline{~}end\underline{~}catch()函数。要打印错误消息,将使用C运行时库的puts()函数,而要从异常中获取类型信息,必须生成对llvm.eh.typeid.for指令的调用。所有的函数都需要FunctionType和Function实例,我们使用createFunc()方法进行创建:
\begin{lstlisting}[caption={}]
void addLandingPad() {
	FunctionType *TypeIdFty; Function *TypeIdFn;
	createFunc(TypeIdFty, TypeIdFn,
				"llvm.eh.typeid.for", Int32Ty,
				{Int8PtrTy});
	FunctionType *BeginCatchFty; Function 
		*BeginCatchFn;
	createFunc(BeginCatchFty, BeginCatchFn,
				"__cxa_begin_catch", Int8PtrTy,
				{Int8PtrTy});
	FunctionType *EndCatchFty; Function *EndCatchFn;
	createFunc(EndCatchFty, EndCatchFn,
				"__cxa_end_catch", VoidTy);
	FunctionType *PutsFty; Function *PutsFn;
	createFunc(PutsFty, PutsFn, "puts", Int32Ty,
				{Int8PtrTy});
\end{lstlisting}

\item 着陆指令是我们生成的第一条指令。结果类型是一个包含i8*和i32类型字段的结构。这个结构是通过调用StructType::get()函数生成的。我们处理一个C++的int类型的异常,必须将this作为子句添加到landingpad指令中。子句必须是i8*类型的常量,因此需要生成一个bitcast指令将TypeInfo值转换为这种类型。将指令返回的值存储在Exc变量中,以备以后使用:
\begin{lstlisting}[caption={}]
	LandingPadInst *Exc = Builder.CreateLandingPad(
		StructType::get(Int8PtrTy, Int32Ty), 1, "exc");
	Exc->addClause(ConstantExpr::getBitCast(TypeInfo, 
			Int8PtrTy));
\end{lstlisting}

\item 接下来,从返回值中提取类型选择器。通过llvm.eh.typeid.for指令,我们获取TypeInfo字段的类型ID,其表示为C++的int类型。通过这个IR,我们现在已经生成了两个需要比较的值,以确定是否可以处理异常:
\begin{lstlisting}[caption={}]
	Value *Sel = Builder.CreateExtractValue(Exc, {1}, 
				"exc.sel");
	CallInst *Id =
		Builder.CreateCall(TypeIdFty, TypeIdFn,
							{ConstantExpr::getBitCast(
								TypeInfo, Int8PtrTy)});
\end{lstlisting}

\item 为了生成用于比较的IR,调用createICmpEq()。这个函数还生成两个基本块,存储在TrueDest和FalseDest变量中:
\begin{lstlisting}[caption={}]
	BasicBlock *TrueDest, *FalseDest;
	createICmpEq(Sel, Id, TrueDest, FalseDest, 
				"match",
				"resume");
\end{lstlisting}

\item 如果这两个值不匹配,控制流将继续在FalseDest基本块上运行。这个基本块只包含一个resume指令,用于将控制权交还给C++运行时:
\begin{lstlisting}[caption={}]
	Builder.SetInsertPoint(FalseDest);
	Builder.CreateResume(Exc);
\end{lstlisting}

\item 如果这两个值相等,控制流将继续在TrueDest基本块上运行。首先生成IR代码,从存储在Exc变量中的landingpad指令的返回值中提取指向异常的指针。然后,生成对\underline{~}cxa\underline{~}begin\underline{~}catch()函数的调用,将异常的指针作为参数传递,这表明运行时开始处理异常:
\begin{lstlisting}[caption={}]
	Builder.SetInsertPoint(TrueDest);
	Value *Ptr =
		Builder.CreateExtractValue(Exc, {0}, 
			"exc.ptr");
	Builder.CreateCall(BeginCatchFty, BeginCatchFn,
						{Ptr});
\end{lstlisting}

\item 通过调用puts()函数来处理异常,将消息打印到控制台。为此,首先通过调用CreateGlobalString\allowbreak Ptr()函数生成一个指向该字符串的指针,然后在生成的调用puts()函数中将该指针作为参数传入:
\begin{lstlisting}[caption={}]
	Builder.CreateCall(EndCatchFty, EndCatchFn);
	Builder.CreateRet(Int32Zero);
}
\end{lstlisting}
\end{enumerate}

通过addThrow()和addLandingPad()函数,可以生成IR来触发异常并处理异常。我们仍然需要添加IR来检查除数是否为0,这是下一节的主题。\par

\hspace*{\fill} \par %插入空行
\textbf{将异常处理代码集成到应用程序中}

除法的IR是在visit(BinaryOp\&)中生成的。我们不只是生成一个sdiv指令,而是首先生成IR来将除数与0进行比较。如果除数为0,则控制流继续在基本块中引发异常。否则,控制流将继续使用sdiv指令在基本块中运行。在createICmpEq()和addThrow()函数的帮助下,我们可以很容易地编写代码:\par

\begin{lstlisting}[caption={}]
	case BinaryOp::Div:
		BasicBlock *TrueDest, *FalseDest;
		createICmpEq(Right, Int32Zero, TrueDest,
					 FalseDest, "divbyzero", "notzero");
		Builder.SetInsertPoint(TrueDest);
		addThrow(42); // Arbitrary payload value.
		Builder.SetInsertPoint(FalseDest);
		V = Builder.CreateSDiv(Left, Right);
		break;
\end{lstlisting}

代码生成部分现在已经完成。为了构建应用程序,可以切换到build目录并运行ninja,如下所示:\par

\begin{tcolorbox}[colback=white,colframe=black]
\$ ninja
\end{tcolorbox}

构建完成后,您可以检查生成的IR——例如,使用with a: 3/a表达式,如下所示:\par

\begin{tcolorbox}[colback=white,colframe=black]
\$ src/calc "with a: 3/a"
\end{tcolorbox}

您将看到引发和捕获异常所需的IR。\par

生成的IR现在依赖于C++运行时。链接所需库的最简单方法是使用clang++编译器,将rtcalc.c文件重命名为rtcalc.cpp表达式计算器的运行时函数,并在文件中的每个函数前面添加extern "C"。然后可以使用llc工具将生成的IR转换为一个目标文件,并使用clang++编译器创建一个可执行文件:\par

\begin{tcolorbox}[colback=white,colframe=black]
\$ src/calc "with a: 3/a" | llc -filetype obj -o exp.o \\
\$ clang++ -o exp exp.o ../rtcalc.cpp
\end{tcolorbox}

然后,可以用不同的值运行应用:\par

\begin{tcolorbox}[colback=white,colframe=black]
\$ ./exp \\
Enter a value for a: 1 \\
The result is: 3\\
\\
\$ ./exp \\
Enter a value for a: 0 \\
Divide by zero!
\end{tcolorbox}

第二次运行中,输入为0,这将引发一个异常。能够正常工作!\par

我们已经学习了如何引发和捕获异常,生成IR的代码可以用作其他编译器的蓝图。当然,所使用的类型信息和catch子句的数量取决于编译器的输入,不过我们需要生成的IR遵循本节中的模式。\par

添加元数据是向LLVM提供进一步信息的一种方法。下一节中,我们将添加类型元数据,并在某些情况下支持LLVM优化器。\par










