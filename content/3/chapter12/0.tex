LLVM has a very flexible architecture. You can also add a new target backend to it. The core of a backend is the target description, from which most of the code is generated. However, it is not yet possible to generate a complete backend, and implementing the calling convention requires manually written code. In this chapter, we will learn how to add support for a historical CPU.\par

In this chapter, we will cover the following:\par

\begin{itemize}
\item Setting the stage for a new backend introduces you to the M88k CPU architecture and shows you where to find the information you need.

\item Adding the new architecture to the Triple class teaches you how to make LLVM aware of a new CPU architecture.

\item In Extending the ELF file format definition in LLVM, you will add support for the M88k-specific relocations to the libraries and tools that handle ELD object files.

\item In Creating the target description, you will develop all the parts of the target description in the TableGen language.

\item In Implementing the DAG instruction selection classes, you will create the passes and supporting classes required for instruction selection.

\item Generating assembler instructions teaches you how to implement the assembler printer, which is responsible for textual assembler generation.

\item In Emitting machine code, you learn about which additional classes you must provide to enable the machine code (MC) layer to write code to object files.

\item In Adding support for disassembling, you will learn how to implement support for a disassembler.

\item In Piecing it all together, you will integrate the source for the new backend into the build system.
\end{itemize}

By the end of this chapter, you will know how to develop a new and complete backend. You will know about the different parts a backend is made of, giving you a deeper understanding of the LLVM architecture.\par



















